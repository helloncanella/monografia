\chapter*[Introdução]{Introdução}
\addcontentsline{toc}{chapter}{Introdução}

Segundo PISA de 2015\footnote{PISA, sigla do \textit{Programme for International Student Assessment} -- Programa internacional para avaliação de alunos, em tradução livre -- é uma proposta de avaliação conduzido pela OCDE (Organização para a Cooperação e Desenvolvimento econômico)}, apenas $30,6\%$ dos alunos brasileiros alcançaram um patamar de competência básico na área de ciências, necessário para lidar com demandas simples da vida cotidiana, segundo o critério dos avaliadores. Este resultado coloca o Brasil na posição 63 entre os 70 países onde o teste foi aplicado.

De modo geral a relação entre estudantes e as disciplinas científicas não é apenas marcado por dificuldades conceituas, mas também por aquelas que envolvem elaboração de estratégias para a resolução de problemas, comuns na atividade científica.


