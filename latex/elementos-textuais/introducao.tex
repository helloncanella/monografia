\chapter*[Introdução]{Introdução}
\addcontentsline{toc}{chapter}{Introdução}

Segundo PISA de 2015\footnote{PISA, sigla do \textit{Programme for International Student Assessment} -- Programa internacional para avaliação de alunos, em tradução livre -- é uma proposta de avaliação conduzido pela OCDE (Organização para a Cooperação e Desenvolvimento econômico)}, apenas $30,6\%$ dos alunos brasileiros alcançaram um patamar de competência básico na área de ciências, necessário para lidar com demandas simples da vida cotidiana, segundo o critério dos avaliadores. Este resultado coloca o Brasil na posição 63 entre os 70 países onde o teste foi aplicado.

De modo geral a relação entre estudantes e as disciplinas científicas não tem sido marcada apenas por dificuldades conceituais, mas também por aquelas que envolvem elaboração de estratégias para a resolução de problemas, comuns na atividade científica. Dentre as dificuldades comuns demonstradas por alunos nas aulas ciências, pode-se destacar \cite{Pozo}:


\begin{enumerate}
  \item Dificuldade para transpor a contextos novos os conceitos e estratégias de resolução de problemas aprendidos. Quando o formato do problema muda os alunos apresentam dificuldades para aplicar à nova situação os conceitos e algoritmos adquiridos. 

  \item Escassez de significado dos resultados obtidos. Ao lidar com problemas propostos em sala de aula, eles se limitam a buscar uma fórmula para chegar a um resultado numérico. Aplicam, sem exercício crítico, o algoritmo de um modelo de problema, visando uma resposta ``correta'' e única. Desse modo, a aplicação de uma técnica se impõe à necessidade de compreender uma estratégia e seus potenciais casos de uso. Convertem, assim, o problema em um simples exercício autômato. 
  
  \item Ausência de motivação para aprendizagem de temas científicos.
\end{enumerate}

Se atentarmos como detalhe essas dificuldades, teremos facilidade em perceber que elas tem sua origem na forma como conteúdos científicos são apresentados em sala de aula, e por ela são induzidas.

Em física, por exemplo, é comum a proposição de problemas cuja solução é alcançada por um procedimento ou rotina automatizável (``Qual é velocidade após $32s$ de um objeto lançado verticalmente como uma velocidade de $20m/s$?''). 

A ausência de motivação, por sua vez, tem sua origem numa percepção de ausência de significado dos temas apresentados, estimulada também pelos mesmos problemas propostos. Qual é apelo e capacidade mobilizadora de algo ``sem significado'' e, consequentemente, ``sem utilidade''? 

Cabe aqui uma observação de \citeonline{Pozo},

\begin{citacao}
Os alunos não aprendem por não estar motivados, mas, por sua vez, não estão motivados porque não aprendem. A motivação não é uma responsabilidade dos alunos, mas também resultado da educação que recebem e, no nosso caso, de como lhes a ciência lhes é ensinada \cite{Pozo}.
\end{citacao}

Afinal, como o ensino de ciências pode ser desenhado de forma a atacar o problema da motivação? Segundo a bem colocada definição de \apudonline{Claxton}{Pozo}, ``motivar é mudar as prioridades de uma pessoa'', tomando como partida os seus próprios interesses para a construção de outros novos. Nesse sentido, o papel do ensino de ciência corresponderia à busca de envolvimento com os problemas pertinentes à realidade do aluno e a seus centros de interesse como meio de estímulo para o seu engajamento -- componente essencial para a construção de aprendizagem significativa. 

É justamente nessa premissa que esse trabalho se sustentará. Nele apresentamos e desdobramos um conceito que vem ganhando relevância nos meios educacionais, que tem como ponto de partida um conjunto de práticas de resolução de problemas há décadas utilizados por programadores e cientistas da computação, conhecido como \textbf{pensamento computacional}.

Em essência, a adoção de uma abordagem ``computacional'' corresponde justamente ao aprendizado e uso de formas de tratar problemas, de qualquer natureza, de modo a tornar possível a sua resolução por computadores.




% Nele iremos discutir como discutir as possibilidades que o conceito de ``Pensamento Computacional''



% O mesmo autor nos aponta a direção.





% A nossa resposta se resume na palavra engajamento. Ou seja

% É justamente

% É justamente com o prósito de aprentar 

% Tendo como ponto de partida a necessidade de apresentar uma solução para falta de relevância para os alunos que esse trabalho concentrará os seu esforços.


% de significação falta 

% É nesse aspecto da aprendizagem que este trabalho concetrará


% ``Os alunos não aprendem porque não estão motivados''

% Los alumnos no aprenden porque no están motivados, pero a su vez no están motivados porque no aprenden. La motivación no es ya sólo una respon-sabilidad de los alumnos (que sigue siéndolo) sino también un resultado de la educación que reciben y, en nuestro caso, de cómo se les enseña la ciencia.



% , um motor definidor de prioridade 












