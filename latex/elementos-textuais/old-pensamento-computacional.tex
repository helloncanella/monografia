\chapter{Pensamento Computacional}\label{pensamento-computacional}

\section{Histórico e definição}

A evolução da computação nos últimos anos, aliado ao seu barateamento, tem tornado possível o desenvolvimento de novas estratégias para resolução de problemas. Fato que tem produzido impactos notáveis tanto na academia quanto nas empresas.

O aprofundamento do entendimento de sistemas orgânicos, por exemplo, trazido pelo aperfeiçoamento dos métodos de modelagem computacional, tanto quanto a coleta e a mineração massiva de dados pelas empresas do varejo com o fim de investigar comportamentos de consumo, ilustram esse fenômeno.

Essencialmente, o que o desenvolvimento recente da computação tem facultado é a otimização de métodos que permitem a exploração de modelos dinâmicos, ou não-determinísticos, sejam eles o avanço de doenças, aspectos da mudança climática ou simplesmente cliques ou curtidas em redes sociais \cite{weintrop}. 

Alguns desdobramentos desse fenômeno também são visíveis no nosso cotidiano. Ao utilizarmos um mecanismo de busca, ou quando utilizamos um tradutor, estamos nos apoiando em algoritmos de coleta e análises de dados que dotam computadores, literalmente, com a capacidade aprender -- processo conhecido como aprendizado de máquina (em inglês: \textit{machine learning}). Algumas projeções revelam que em poucas décadas um número expressivo de atividades laborais serão simplesmente substituídas por inteligência artificial. %TODO: Procurar referência

No ano de 2006, refletindo sobre as novas práticas de resolução de problemas trazidas pela evolução da capacidade e dos métodos computacionais, a professora da Universidade Carnegie Mellon, Jeannette Wing, em um artigo seminal, elabora o conceito de \textbf{pensamento computacional}. 

O uso dessa expressão, contudo, não é nova. O seu primeiro registro pode ser encontrado no livro \textit{Mindstorms: Children, computers and powerful ideas}, da década de 1980, onde Seymour Papert discorre sobre a oportunidade trazida por computadores para o ensino de matemática. A sua popularização, todavia, se dá apenas a partir da publicação do artigo por \citeonline{wing2006}.

Neste trabalho, a autora descreve o pensamento computacional como uma ``abordagem para a solução de problemas, desenho de sistemas e entendimento do comportamento humano que se vale de conceitos fundamentais para ciência da computação'' \cite[tradução nossa]{wing2006}.

Trata-se de um procedimento analítico que compartilha semelhanças com a matemática na forma resolve problemas a partir da criação de abstrações, e com a engenharia no modo como aborda a criação e a avaliação de sistemas grandes e complexos, próprios da realidade física \cite[p.~3717]{wing2008}.

O elemento-chave do pensamento computacional é a abstração. Contudo, diferentemente do pensamento matemático, a composição da abstração nesse caso deve levar em conta os casos-limite da realidade física. O que acontece, por exemplo, quando a capacidade do disco rígido atinge o limite? Quais são as ações a serem tomadas quando o um carro auto-dirigível ``avista'' um obstáculo? Esses são os tipos de questões a serem respondidas quando problemas são resolvidos por abstrações computacionais \cite[p.~3718]{wing2008}.

% Abstraction - reduce a problem to its bare essence.

O processo de abstração envolve a definição do seu nível de detalhamento: a escolha dos aspectos a serem destacados, aquilo que podemos considerar importante para descrição do problema e, consequentemente, o descarte dos fatores que julgamos irrelevantes. 

Esse procedimento implicará necessariamente a representação do problema em unidades menores, a sua decomposição em camadas, bem como a especificação das relações entre elas, que poderão então ser automatizadas por métodos computacionais \cite[p.~3718]{wing2008}. A automação, por seu turno, cumprirá o papel de permitir a representação dessas interações e de suas consequências em escalas maiores. %(falar mais a respeito)

% falar sobre o uso de linguagem de programação para automação.

% fazer mais reflexões sobre abstração. Discutir o seu papel de representar casos particulares.

% discutir recursos como "paralelismo", "modulação", "parametrização" e "recursividade" como formas de construir abstrações e as relações entre elas. (Ver RESNICK)

% discutir algoritmos podem ser vistos como abstrações de um conjunto de passos para execução de uma tarefa.

% discutir como a construção de abstrações podem ser feitas de modo recursivo, favorecendo a criação modelos/sistemas complexos.

Ao fazermos uso da expressão `métodos computacionais' ou, simplesmente, `computação', alguns esclarecimentos são necessários. Como destaca \citeonline{wing2008}, a conceituação que envolve esses termos extrapola a imagem óbvia do dispositivo físico clássicos com processamento e capacidade de armazenamento. Se atentarmos para o fato que seres humanos também calculam e computam, e que são ainda melhores que máquinas para atividades como reconhecimento de imagens, por exemplo, poderemos generalizar o uso dessas palavras para nos referirmos à combinação homem-máquina, ou recursivamente, a uma rede composta por vários seres humanos e máquinas\footnote{Podemos ainda ampliar o conjunto dessas `máquinas' para além dos dispositivos elétricos que temos nas nossas mesas e nas nossas mãos, se incluirmos novas tecnologias tais como os computadores biológicos. Para mais informações leia \citeonline{biocomputer}.} \cite{wing2008,Wing2010}.

A multiplicação dos contextos onde o pensamento computacional vem sendo aplicados, estimulados pelas contínuas inovações trazidas por ele, traz consigo várias questões educacionais. Afinal, se computadores estão em toda parte, como habilitar os indivíduos a fazerem uso dessas ferramentas de modo que possam criar valor para si? Como permitir que eles invertam o papel de meros consumidores de tecnologia para então também se tornarem criadores? Dotá-los com a capacidade de pensar computacionalmente significará permitir que eles possam descrever e resolver problemas de modo a admitir solução computacional \cite{Wing2010}. 

% A multiplicação dos contextos onde o pensamento computacional vem sendo aplicados, estimulados pelas contínuas inovações trazidas por ele, desencadearão alguns fenômenos pertinentes à educação. As pressões culturais e econômicas gerados por esse processo irão colaborar para que estudantes tenha contato com o tema cada vez mais cedo. 

% No mais, essa difusão traz consigo muitas outras questões educacionais. Afinal, se computadores estão em toda parte, como habilitar os indivíduos a fazerem uso dessas ferramentas de modo que possam criar valor para si? Como permitir que eles invertam o papel de meros consumidores de tecnologia para então também se tornarem criadores? Dotá-los com a capacidade de pensar computacionalmente significará permitir que eles possam descrever e resolver problemas de modo a admitir solução computacional \cite{Wing2010}. 

% Além disso, como consequência natural das pressões culturais e econômicas trazidas onipresença dos computadores, é provável que estudantes tenham contato com o tema cada vez mais cedo. Por esse motivo que \citeonline{wing2008} acredita que, progressivamente, o pensamento computacional será parte integrante do currículos escolares. Mesmo onde os contextos políticos 

% No mais, a intensificação das inovações trazidas pelo pensamento computacional, favorecidas, por exemplo, pela capacidade progressiva de computadores analisarem volumes astronômicos de dados, fará com que esse tema ocupe gradativamente todas as areas da atividade humana. Com a educação não será diferente. As pressões decorrentes desse fato propiciará o contato de estudantes com assunto, inclusive, nas primeiras séries. 

% "É por esse motivo que \citeonline{wing2006} acredita que o pensamento computacional será incorporado gradativamente as competências analíticas de todo o estudante". %MELHORAR!!!!!!!

% Dentre as várias questões propostas por \citeonline{wing2008}, podemos citar: quais são as melhores formas de ensinar o pensamento computacional?  

% retomar os últimos parágrafos, destacando como os processos descritos resumem o pensamento computacional.

% Em linhas gerais, o conceito descreve três etapas para a resolução de problemas: % \begin{enumerate} \item Descrição do problema a partir da criação e da articulação de abstrações. \item \end{enumerate}

% A idea de um "pensamento" computacional, busca, no entender da autora, abranger um conjunto de habilidades

% TODO: trazer elementos da ciência da computação para discussão, mostrando o seu potencial para resolver problemas (usar tabela comparativa se possível).

% TODO: para reforçar a importância da ciência da computação nas mais diversas áreas, citar a adoção de curso de programação em cursos de graduação diferentes da ciência computação.

% TODO: citar como exemplo da relevância da forma de pensar computacionalmente, a contratação em massa de mestres e doutores Ciência computação em wall street. Usar https://insights.dice.com/2017/11/20/got-phd-cs-statistics-consider-wall-street/ como referência 

% TODO: citar exemplos apresentados pela autora de possíveis usos de um "pensamento" computacional em tarefas do dia-dia (Trata-se de uma argumentação fraca, mas legítima. Apenas considerar.)

% Conforme sintetiza em um outro artigo, no ano de 2010,
% \begin{citacao} O pensamento computacional $[$...$]$ descreve a atividade mental na formulação de um problema para admitir uma solução computacional. A solução pode ser realizada por um humano ou máquina, ou mais geralmente, por combinações de seres humanos e máquinas \cite[tradução nossa]{Wing2010}. \end{citacao} 


% Para a autora, a ideia de uma "pensamento" ou simplesmente de uma "abordagem" computacional para resolução problemas, não seria apenas relevante para programadores ou cientistas da computação, 

% Em essência, o que se busca compor uma abordagem para resolução de problemas, este conceito e propõe a abranger um conjunto de habilidades e práticas não apenas relevante para programadores ou cientistas da computação, mas passível de compor até mesmo as faculdades analíticas de qualquer criança \cite{wing2006}.

% O uso da palavra "computacional" busca também evidenciar a relação dessa abordagem % com elementos e temas próprios da ciência da computação, tais como a confecção de algorítimos (redação de passos para resolver um problemas), 