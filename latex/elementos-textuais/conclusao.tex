% ----------------------------------------------------------
% Conclusão (outro exemplo de capítulo sem numeração e presente no sumário)
% ----------------------------------------------------------
\chapter*[Conclusão]{Conclusão}
\addcontentsline{toc}{chapter}{Conclusão}
% ----------------------------------------------------------

% \lipsum[31-33]
A pesquisa necessária para o desenvolvimento deste trabalho, tanto quanto a sua produção, foi de especial importância para o autor refletir e ampliar seus conhecimentos sobre os diversos aspectos que envolvem a aprendizagem científica. 

A relevância do tema para a formação educacional da sociedade e, em última análise, para seu progresso material exigem pesquisas, reflexões e debates contínuos, principalmente em face dos problemas de baixo de desempenho e falta de motivação dos estudantes brasileiros com relação a temas científicos, evidentes aos olhos de professores veteranos, e continuamente quantificados por testes nacionais e internacionais. 

A construção desse trabalho foi orientada pela expectativa de trazer um conjunto de reflexões e de propostas que tornem o aprendizado de temas científicos, e das metodologias que lhe servem de apoio, mais atrativo ao estudante. 

Com esse propósito, trouxemos à discussão o conceito de pensamento computacional e vimos como a sua adoção em sala de aula favorece a construção de um contexto ótimo para o desenvolvimento de competências científicas. Como explicado, esse fato se justifica pelos elementos que compartilha com um largo espectro de disciplinas científicas, tal como as noções de abstração e decomposição de problemas.

Como indicado pela literatura, a percepção de estar resolvendo um problema científico autêntico e que, em alguns casos, dialogue com sua realidade, tem se mostrado capaz de aumentar os níveis de engajamento e interesse dos alunos -- fatores cruciais para a aprendizagem significativa -- e assim lidar com o problema da motivação com o qual os professores estão habituados.

Além disso, o estímulo à construção de modelos e o convite à reflexão crítica sobre os seus limites de validade, trazidos por essa prática, propicia a formação de um novo tipo de relação entre o estudante e as disciplinas científicas, que substitui o seu papel de apenas de reproduzir o que lhe foi transmitido, durante uma avaliação.

Como pudemos discutir, apesar de haver consenso sobre os princípios e elementos que compõe essa conceituação, o mesmo não ocorre com relação às diretrizes e estratégias para sua aplicação em sala de aula. Mesmo que o seu potencial pedagógico tenha sido identificado, podemos ainda observar na literatura, por exemplo, percepções difusas sobre quais competências cognitivas e científicas a serem desenvolvidas quando se pretende fazer com que os alunos pensem ``computacionalmente''. 

A contribuição a ser dada, objetivada nessa monografia, é a repercussão do trabalho de \citeonline{Weintrop2016} que define com clareza tais competências e a demonstração do como a taxonomia que as incorpora pode ser útil na construção de atividades com conteúdo cientificamente relevante.


% Além da discussão sobre os elementos básicos que compõe essa noção, tivemos a oportunidade de aprofundar Tivemos a oportunidade 

% Neste trabalho pudemos discutir alguns 



% Apesar de olhar

% Apesar desses problemas não terem sido objetos de analise ao longo desse trabalho blalablala....


% Roteiro de consêquencias

% - excelente contexto para o desenvolvimento de competências científicas, já que compartilha elementos comuns com disciplinas científicas tais compo a noção de abstração e decomposição de problemas

% - problemas matemáticos são excelente domínios para aplicação do pensamento computacional

% - o uso do computador que agrega diversidade e flexibilidades nas abordagens para tratar problemas.

% -  sensação de estar resolvendo um problema autêntico com realaplicabilidade e, em muitos casos, inserido na realidade do qual fazem parte, produzaumentos sensíveis dos níveis de engajamento

% - o uso do computador permite que alunos resolvam problemas cientítificos autênticos o que colabora para o ganho de relevancia de ciencieas ao seus olhos, o que aumenta atratividade das disciplinas cientificas

% - familirizar os alunos com a noção de modelo, co entidade representativa, que antes de oferecer verdades absolutos, como tentam fazer acreditar os professores, é capaz de oferecer apenas algumas respostas. 

% - estimula a inversão  dos alunos com a tecnologia, marcada apenas pelo seu consumo  e não pela sua produção.

% - NOTA: descrever a relação entre alunos treinados por atividades relaciona

% - substitui uma relação carecterizada pela reprodução autmata 
