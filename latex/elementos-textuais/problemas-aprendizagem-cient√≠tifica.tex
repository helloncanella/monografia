PROBLEMAS

Segundo o PISA 2006, mais de 60\% dos alunos brasileiros não apresentam competência suficiente na área de Ciências para lidar com as exigências e os desafios mais simples da vida cotidiana atual. Resultado: o Brasil ocupa o lamentável 52o lugar entre os 57 países submetidos ao exame. Sem erradicar seu “analfabetismo científico”, dificilmente o Brasil conseguirá atingir a meta do Ministério da Educação, contida em seu Plano de Desenvolvimento da Educação (PDE), de alcançar, até 2022, o nível que hoje ostentam os países mais industrializados, membros da OCDE. Cumpre lembrar, porém, que as posições desses países no ranking do PISA, em 2022, também deverão elevar-se.. 


(O PISA, sigla do Programme for International Student Assessment – Programa Internacional para a Avaliação de Alunos –, é uma proposta de avaliação promovida pela OCDE (Organização para a Cooperação e o Desenvolvimento Econômico), uma entidade intergovernamental dos países industrializados que atua em modo de foro de promoção do desenvolvimento econômico e social dos membros. Nas avaliações, além desses membros, tomam parte também outros não pertencentes, que atuam sob o rótulo de países convidados. A avaliação de 2006, foco do presente trabalho, teve a participação de 30 países membros da OCDE e de 27 convidados. Desses 57 países, seis eram latino-americanos: Argentina, Brasil, Chile, Colômbia, México e Uruguai, sendo México o único desse grupo que é membro da OCDE. )

-----

Pero los alumnos no sólo encuentran dificultades conceptuales, también las tienen en el uso de estrategias de razonamiento y solución de pro- blemas propios del trabajo científico. La Tabla 1.2 resume algunas de las dificul- tades más comunes en el dominio de lo que podemos llamar los contenidos procedimentales del currículo de ciencias, lo que tienen que aprender a hacer con sus conocimientos científicos.

-------

Algunas dificultades en el aprendizaje de procedimientos en el caso de los problemas cuantitativos. (Extraído de POZO y GÓMEZ CRESPO, 1996). (page 17)

1.- Escasa generalización de los procedimientos adquiridos a otros contextos nuevos. En cuanto el formato o el contenido conceptual del problema cambia, los alumnos se sienten incapaces de aplicar a esa nueva situación los algoritmos aprendidos. El verdadero problema de los alumnos es saber de qué va el proble- ma (de regla de tres, de equilibrio químico, etc).

2.- El escaso significado que tiene el resultado obtenido para los alumnos. Por lo general, aparecen superpuestos dos problemas, el de ciencias y el de matemáti- cas, de forma que, en muchas ocasiones este último enmascara al primero. Los alumnos se limitan a encontrar la “fórmula” matemática y llegar a un resultado numérico, olvidando el problema de ciencias. Aplican ciegamente un algoritmo o un modelo de “problema” sin comprender lo que hacen.

3.- Escaso control metacognitivo alcanzado por los alumnos sobre sus propios procesos de solución. La tarea se ve reducida a la identificación del tipo de ejer- cicio, y a seguir de forma algorítmica los pasos que ha seguido en ejercicios simi- lares en busca de la solución “correcta” (normalmente única). El alumno apenas se fija en el proceso, sólo le interesa el resultado (que es lo que suele evaluarse). De esta forma, la técnica se impone sobre la estrategia y el problema se convierte en un simple ejercicio rutinario.

4.- El escaso interés que esos problemas despiertan en los alumnos, cuando se utilizan de forma masiva y descontextualizada, reduciendo su motivación para el aprendizaje de la ciencia.

------

En el Capítulo III analizaremos con detalle estas dificul-tades de aprendizaje y sus posibles soluciones, pero sin duda buena parte de ellas se deben a las propias prácticas escolares en solución de problemas, que tienden a centrarse más en tareas rutinarias o cerradas, con escaso signi-ficado científico (“cuál será la velocidad alcanzada a los 43 segundos por un proyectil que, partiendo del reposo, está sometido a una aceleración constan-te de 2 m/s2?”), que en verdaderos problemas con contenido científico (“¿por qué son los días más largos en verano que en invierno?”).

-----

Algunas actitudes y creencias inadecuadas mantenidas por los alumnos con respecto a la naturaleza de la ciencia y a su aprendizaje (page 18)

- Aprender ciencia consiste en repetir de la mejor forma posible lo que explica el pro- fesor en clase
- Para aprender ciencia es mejor no intentar encontrar tus propias respuestas sino aceptar lo que dice el profesor y el libro de texto, ya que está basado en el conoci- miento científico
- El conocimiento científico es muy útil para trabajar en el laboratorio, para investigar y para inventar cosas nuevas, pero apenas sirve para nada en la vida cotidiana
- La ciencia nos proporciona un conocimiento verdadero y aceptado por todos - Cuando sobre un mismo hecho hay dos teorías, es que una de ellas es falsa: la ciencia acabará demostrando cuál de ellas es la verdadera
- El conocimiento científico es siempre neutro y objetivo - Los científicos son personas muy inteligentes, pero un tanto raras, que viven ence- rrados en su laboratorio
- El conocimiento científico está en el origen de todos los descubrimientos tecnológi- cos y acabará por sustituir a todas las demás formas del saber
- El conocimiento científico trae consigo siempre una mejora en la forma de vida de la gente

% ------

% La distribu-ción de premios y castigos es sin duda un mecanismo eficaz para controlar la conducta de los alumnos, pero, como veremos más adelante al tratar el proble-ma de la motivación, es un sistema en sí mismo limitado para lograr cambios estables y duraderos en las actitudes de los alumnos. Debe acompañarse por otros mecanismos específicos de aprendizaje social.

----

A forma como a ciência é aprendida colabora com a imagem que os aluos formam dela.

-----

De hecho, reducir la “actitud científica” a la aplicación ciega de unos procedimien-tos preestablecidos es lo opuesto del espíritu de curiosidad, indagación y auto-nomía que deben caracterizar al hacer científico. La enseñanza del mal llamado “método científico” en lugar de promover hábitos propios del trabajo científico suele ahogar las verdaderas actitudes científicas que tímidamente puedan mani-festar los alumnos.

----

Cuando se trata de identificar los presuntos culpa-bles de la falta de aprendizaje de la ciencia, sobre todo en educación secunda-ria, la mayoría de los profesores adoptarían como sospechoso número uno a la motivación, o para ser precisos, la falta de motivación de sus alumnos, sin duda el enemigo público número uno de la enseñanza de la ciencia.

----

Sin duda es un diagnóstico certero ya que la motivación es uno de los problemas más graves del aprendizaje en casi todas las áreas, no sólo en ciencias. Durante la educación obligatoria, coincidiendo con la adolescencia, es cuando los alumnos, debido a su propio desarrollo personal, comienzan a fijarse sus propias metas, a establecer sus preferencias y a adoptar actitudes que no siem- pre favorecen el aprendizaje.

-----

La investigación psicológica ha mostrado la impor-tancia de la motivación en el aprendizaje. Sin motivación no hay aprendizaje escolar. Dado que el aprendizaje, al menos el explícito e intencional, requiere continuidad, práctica, esfuerzo, es necesario tener motivos para esforzarse, es necesario (en la etimología de la palabra motivación) moverse hacia el aprendi-zaje. ¿Tienen los alumnos adolescentes motivos para esforzarse en aprender ciencias? ¿Es la motivación sólo un problema de los alumnos? ¿Son ellos los que no tienen motivos para aprender o es la propia enseñanza la que no les mueve a aprender?

---

En este modelo, la motivación es una responsabili-dad sólo de los alumnos, debida a su falta de interés por el conocimiento, el esfuerzo intelectual o la educación en general, a la que conceden escaso valor. Aunque estos rasgos puedan ser en algunos casos válidos, la motivación debe concebirse de forma más compleja, no sólo como una causa de la falta de aprendizaje de la ciencia, sino también como una de sus primeras consecuen-cias. Los alumnos no aprenden porque no están motivados, pero a su vez no están motivados porque no aprenden. La motivación no es ya sólo una respon-sabilidad de los alumnos (que sigue siéndolo) sino también un resultado de la educación que reciben y, en nuestro caso, de cómo se les enseña la ciencia.

--

Normalmente no es que no estén motivados, que no se muevan en absoluto, sino que se mueven para cosas diferentes y en direcciones distin-tas a las que pretenden sus profesores. En este sentido dice CLAXTON (1984) que motivar es cambiar las prioridades de una persona, sus actitudes ante el aprendizaje. No podemos dar por supuesto de antemano que los alumnos estén interesados por aprender ciencia. Uno de los objetivos de la educación científica debe ser precisamente despertar en ellos ese interés.

----

[SOBRE RAZÕES EXTRINSICAS A CIENCIA QUE MOBILIZAM O ALUNO A APRENDE-LA] Obviamente los premios y castigos que mueven a los alumnos son más sutiles y complejos, ya que no responden a una necesidad primaria, sino a un deseo socialmente definido (el aprobado, el reconocimiento social, la autoestima, etc.).

Si la conducta aprendida mediante motivación extrínseca es relevante y eficaz, se utiliza en muchos contextos después de haberla aprendido (por ej., escribir en un orde-nador o hablar inglés), los resultados serán duraderos. Pero si, como sucede con frecuencia en las clases de ciencias, lo que se aprende (sea la meiosis, el equilibrio químico o las funciones logarítmicas) no es percibido por el alumno como algo de interés o significativo, ese aprendizaje resultará muy efímero (poco más allá del examen, si llega) y por tanto muy poco eficaz. A veces no sólo no se consiguen los aprendizajes deseados (que los alumnos entiendan la  meiosis) sino incluso se obtienen también resultados indeseables bastante más duraderos (como aborrecer para siempre las ciencias naturales y sus abstrusos conceptos), en forma de actitudes muy difíciles de modificar después. Dado que una de las metas básicas de la educación científica en la secundaria, den-tro de su orientación formativa y no selectiva, como veíamos en el capítulo ante-rior, debe ser crear un interés por la ciencia como forma de acercarse a los pro-blemas que nos rodean, el remedio puede ser en este caso peor que la enfer-medad: tal vez se consiga un aprendizaje muy superficial y efímero de algunos conocimientos científicos al precio de lograr una aversión profunda y duradera hacia esos mismos conocimientos y su aprendizaje.

-----

É importante que os alunos possam atribuir seus fracassos a fatores modificáveis.

----

SOLUÇÕES

[Sobre metáfora newtoniana para falta de motivação] "Mudar a quantidade de movimento" dos alunos, despertando seu interesse por ciencias.

---------

Pero ¿cómo puede fomentarse este interés intrínseco, por la ciencia en sí a través de la instrucción? Retomando la feliz frase de CLAXTON (1984) según la cual “motivar es cambiar las prioridades de una persona”, se trataría de partir de los intereses y preferencias de los alumnos para generar otros nuevos. Para ello la enseñanza debe tomar como punto de partida los intereses de los alumnos, buscar la conexión con su mundo cotidiano, pero con la finalidad de trascenderlo, de ir más allá, e introducirles, casi sin saberlo, en la tarea científi-ca. No hay que suponer que, para aprender ciencia, los alumnos deben tener desde el comienzo las actitudes y motivos de los científicos, más bien hay que diseñar una enseñanza que genere esas actitudes y motivos. Diversos autores (para una revisión véase ALONSO TAPIA, 1997; HUERTAS, 1997) han destacado que esas estrategias didácticas para la motivación deben basarse en la locali-zación de centros de interés, el trabajo cooperativo, la autonomía y la participa-ción activa de los alumnos, etc., implicando cambios sustanciales en la propia organización de las actividades escolares, mostrando que la motivación no es algo que está o no está en el alumno, sino que es un producto de la interac-ción social en el aula.

-----------

Pero, además de cambiar el valor de las tareas, otra forma de mejorar la motivación, volviendo a la ecuación establecida unas páginas más atrás, es aumentar la expectativa de éxito de los alumnos en las tareas. Como decíamos antes, la motivación no sólo es causa, sino también consecuencia, del aprendi- zaje. Sin aprendizaje tampoco hay motivación. Si, a pesar de esforzarse, el alumno tiene la expectativa de que no va a aprobar o a aprender nada (depen- diendo de sus metas), difícilmente se esforzará.Dado que la valoración que hace el alumno de su expectativa de éxito será muy dependiente de la evalua- ción que reciba del profesor, esa evaluación resulta ser uno de los motores fun- damentales de la motivación. Una evaluación que ayude al alumno a compren- der por qué no aprende, cuáles son sus dificultades de aprendizaje, que le ayude a regular su propio aprendizaje, será un factor esencial de su motivación. Si el alumno recibe pistas sobre qué tiene que hacer la próxima vez para tener más éxito, en lugar de una nota simple y llana, será más probable que se esfuerce. Es importante, a partir de ese valor informativo y reflexivo de la evalua- ción, que el alumno atribuya sus fracasos a factores modificables que pueda controlar (la estrategia de estudio seguida, el esfuerzo realizado, sus conoci- mientos, etc.) en vez de a factores incontrolables o ajenos a él (la suerte, la difi- cultad de la asignatura, su capacidad intelectual, etc.).

-----------

Pero además de ayudarle a interpretar mejor sus éxitos y fracasos, un profe- sor puede facilitar la motivación de sus alumnos también de una forma más sim- ple y directa, haciendo más probable el éxito al adecuar las tareas a las verda- deras capacidades y disposiciones de sus alumnos. Por más ayuda que reciba y por más que valore el éxito en la tarea, es poco probable que el lector se sien- ta motivado para batir el récord de la hora en bicicleta. Pero tal vez sí se sienta suficientemente motivado para participar en una carrera popular. Pero adecuar las tareas a las capacidades y conocimientos previos de los alumnos requiere conocer cuáles son las limitaciones en esas capacidades y conocimientos, que puede ser otra de las causas de las dificultades de aprendizaje de los alumnos. De esta forma, vemos que la motivación, entendida como un proceso de cam- bio de actitudes, está estrechamente vinculada con otras dificultades de apren- dizaje. Una de las formas más directas de incrementar el interés de los alumnos por el aprendizaje de la ciencia es lograr que aprendan más en las clases de ciencias, para lo cual hay que tener en cuenta también las dificultades específi- cas que plantean el aprendizaje de procedimientos y de conceptos científicos, que serán los objetivos de los próximos capítulos.













