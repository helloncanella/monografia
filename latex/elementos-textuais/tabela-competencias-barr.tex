
\begin{table}[!htb]
  \caption{Práticas identificadas no pensamento computacional, segundo \citeonline{Barr2011},  presentes/possíveis na educação básica}
  \label{tab:barr-1}

  \begin{center}
    \begin{tabularx}{\textwidth}{@{}YYYYYY@{}}
      \hline 
      \textbf{Práticas} & \textbf{Matemática} & \textbf{Ciência} & \textbf{Estudos Sociais} & \textbf{Linguagens} \\

      \hline
      \\
      \textit{Coleta de   dados} & Encontrar fonte de dados de um problema, lançando dados ou moedas, por exemplo & Recolhimento de dados de um
      experimento & Estudar estatísticas de batalha ou outras estatísticas populacionais & Fazer uma análise linguística
      de sentenças. \\ \\

      \hline
      \\
      \textit{Análise de dados} & Contagem da ocorrências de lançamentos de moeda ou dados e analisar resultados & Analisar os dados a partir de um experimentar & Identificar tendências nos dados a partir de estatísticas & Identificar padrões para frase de diferentes tipos. \\ \\
      
      \hline
      \\
      \textit{ Representação de dados} & Use histograma, gráfico circular, gráfico de barras... Usar conjuntos, listas, gráficos, etc. & Resumir dados de um experimento & Resumir e representam tendências & Representar padrões de diferentes tipos de sentença \\ \\

      \hline
      \\
      \textit{Decomposição de problemas} & Aplicar ordem de operações em uma
      expressão & Fazer classificação de espécies & - & Escrever um esboço \\ \\ 

      \hline
      \\
      \textit{Abstração} & User variáveis em álgebra; estudar funções algébricas em comparação com as funções na programação & Construir um modelo de uma entidade física & Resumir os fatos; deduzir conclusões a partir de factos & - \\ \\% 
      \hline
    \end{tabularx}
  \end{center}
  \legend{Fonte: Adaptado de \citeonline{Barr2011}}
\end{table}

\begin{table}[!htb]
  \caption{\textit{Continuação} - Práticas identificadas no pensamento computacional, segundo \citeonline{Barr2011}, presentes/possíveis na educação básica}
  \label{tab:barr-2}
  \begin{center}
    \begin{tabularx}{\textwidth}{@{}YYYYYY@{}}
      \hline 
      
      \textbf{Práticas} & \textbf{Matemática} & \textbf{Ciências} & \textbf{Estudos Sociais} & \textbf{Linguagens} \\

      \hline
      \\
      \textit{Procedimentos algorítmicos} & Longa divisão, fatoração & Fazer procedimentos experimentais & - & Escrever instruções \\ \\ % 

      \hline
      \\
      \textit{Automação} & - & Usar \textit{Probware} ou \textit{Origin} & Usar excel & Usar um corretor ortográfico \\ \\ 

      \hline
      \\
      \textit{Paralelização} & Resolver sistema lineares; fazer multiplicação de matrizes & Executar simultaneamente experimentos com diferentes parâmetros & - & - \\ \\

      \hline
      \\
      \textit{Simulação} & Representar graficamente uma função em um plano cartesiano e modificar os valores da variáveis & Simular o movimento de o sistema solar & Jogar Age of Empires; Trilha de Oregon & Fazer uma reencenação de uma história \\ \\
      \hline

    \end{tabularx}
  \end{center}
  \legend{Fonte: Adaptado de \citeonline{Barr2011}}
\end{table}