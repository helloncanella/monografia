% ----------------------------------------------------------
% RESUMOS
% ----------------------------------------------------------

% resumo em português
\setlength{\absparsep}{18pt} % ajusta o espaçamento dos parágrafos do resumo
\begin{resumo}
  Exames coordenados pela a OCDE, em 2015, reportam que apenas $30,6\%$ dos alunos brasileiro demonstram competência básica na área de ciências. Dentre os problemas que marcam a relação deles com as disciplinas científicas, é notório a ausência de motivação e percepção de relevância com respeito a temas, problemas e modelos discutidos em sala de aula. Com o interesse de trazer uma abordagem que lide com esse problema, nos propomos a discutir as implicações e aplicações de um conceito, que vem ganhando popularidade nos meios educacionais, conhecido como Pensamento Computacional. Ao longo desse trabalho, nosso objetivo será de repercutir dados e exemplos da literatura que demonstram como ele tem favorecido a construção de um ambiente propício para a aprendizagem de competências científicas. Ao final, apresentaremos algumas atividades com esse propósito, facilmente reproduzíveis em sala de aula.
%  Segundo a \citeonline[3.1-3.2]{NBR6028:2003}, o resumo deve ressaltar o
%  objetivo, o método, os resultados e as conclusões do documento. A ordem e a extensão
%  destes itens dependem do tipo de resumo (informativo ou indicativo) e do
%  tratamento que cada item recebe no documento original. O resumo deve ser
%  precedido da referência do documento, com exceção do resumo inserido no
%  próprio documento. (\ldots) As palavras-chave devem figurar logo abaixo do
%  resumo, antecedidas da expressão Palavras-chave:, separadas entre si por
%  ponto e finalizadas também por ponto.

 \textbf{Palavras-chaves}: Ensino de ciências. STEM. Pensamento computacional.
\end{resumo}

% resumo em inglês
\begin{resumo}[Abstract]
 \begin{otherlanguage*}{english}
  
  Exams coordinated by the OECD in 2015 report that only $ 30.6 \% $ of Brazilian students demonstrate basic competence in the area of science. Among the problems that mark their relation with the scientific disciplines, the absence of motivation and perception of relevance with respect to the themes, problems and models discussed in the classroom is notorious. With the interest of bringing an approach that deals with this problem, we propose to discuss the implications and applications of a concept, which is gaining popularity in the educational research, known as Computational Thinking. Throughout this work, our objective will be make known some data and examples from the literature that demonstrate how it has favored the construction of an environment conducive to the learning of scientific competences. At the end, we will present some activities for that purpose, easily reproducible in the classroom.
   \vspace{\onelineskip}
 
   \noindent 
   \textbf{Key-words}: Scientific teaching. STEM. Computacional thinking.
 \end{otherlanguage*}
\end{resumo}


