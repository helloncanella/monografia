% ----------------------------------------------------------
% EPÍGRAFE
% ----------------------------------------------------------
\begin{epigrafe}
    \vspace*{\fill}
	\begin{flushright}
		\textit{``Do rigor da ciência}	
		
		
		\textit{…Naquele império, a Arte da Cartografia alcançou tal Perfeição que o mapa de uma única Província ocupava uma cidade inteira, e o mapa do Império uma Província inteira. Com o tempo, estes Mapas Desmedidos não bastaram e os Colégios de Cartógrafos levantaram um Mapa do Império que tinha o Tamanho do Império e coincidia com ele ponto por ponto. Menos Dedicadas ao Estudo da Cartografia, as gerações seguintes decidiram que esse dilatado Mapa era Inútil e não sem Impiedade entregaram-no às Inclemências do sol e dos Invernos. Nos Desertos do Oeste perduram despedaçadas Ruínas do Mapa habitadas por Animais e por Mendigos; em todo o País não há outra relíquia das Disciplinas Geográficas. (Suárez Miranda: Viagens de Varões Prudentes, livro quarto, cap. XIV, 1658.)''}
		
		\textit{Jorge Luis Borges}
	\end{flushright}
\end{epigrafe}

